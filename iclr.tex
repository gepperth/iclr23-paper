% for RL, this strategy removes the need for disjoint sub-tasks
% mention review!!
% initial training anders behandeln, erwähnen!
% Was ist mit den CRL-Bildern, wie werden die behandelt??

\documentclass{article} % For LaTeX2e

\usepackage{iclr2023_conference,times}
\usepackage{comment}
\usepackage[pdftex]{graphicx}


% Optional math commands from https://github.com/goodfeli/dlbook_notation.
%%%%% NEW MATH DEFINITIONS %%%%%

\usepackage{amsmath,amsfonts,bm}

% Mark sections of captions for referring to divisions of figures
\newcommand{\figleft}{{\em (Left)}}
\newcommand{\figcenter}{{\em (Center)}}
\newcommand{\figright}{{\em (Right)}}
\newcommand{\figtop}{{\em (Top)}}
\newcommand{\figbottom}{{\em (Bottom)}}
\newcommand{\captiona}{{\em (a)}}
\newcommand{\captionb}{{\em (b)}}
\newcommand{\captionc}{{\em (c)}}
\newcommand{\captiond}{{\em (d)}}

% Highlight a newly defined term
\newcommand{\newterm}[1]{{\bf #1}}


% Figure reference, lower-case.
\def\figref#1{figure~\ref{#1}}
% Figure reference, capital. For start of sentence
\def\Figref#1{Figure~\ref{#1}}
\def\twofigref#1#2{figures \ref{#1} and \ref{#2}}
\def\quadfigref#1#2#3#4{figures \ref{#1}, \ref{#2}, \ref{#3} and \ref{#4}}
% Section reference, lower-case.
\def\secref#1{section~\ref{#1}}
% Section reference, capital.
\def\Secref#1{Section~\ref{#1}}
% Reference to two sections.
\def\twosecrefs#1#2{sections \ref{#1} and \ref{#2}}
% Reference to three sections.
\def\secrefs#1#2#3{sections \ref{#1}, \ref{#2} and \ref{#3}}
% Reference to an equation, lower-case.
\def\eqref#1{equation~\ref{#1}}
% Reference to an equation, upper case
\def\Eqref#1{Equation~\ref{#1}}
% A raw reference to an equation---avoid using if possible
\def\plaineqref#1{\ref{#1}}
% Reference to a chapter, lower-case.
\def\chapref#1{chapter~\ref{#1}}
% Reference to an equation, upper case.
\def\Chapref#1{Chapter~\ref{#1}}
% Reference to a range of chapters
\def\rangechapref#1#2{chapters\ref{#1}--\ref{#2}}
% Reference to an algorithm, lower-case.
\def\algref#1{algorithm~\ref{#1}}
% Reference to an algorithm, upper case.
\def\Algref#1{Algorithm~\ref{#1}}
\def\twoalgref#1#2{algorithms \ref{#1} and \ref{#2}}
\def\Twoalgref#1#2{Algorithms \ref{#1} and \ref{#2}}
% Reference to a part, lower case
\def\partref#1{part~\ref{#1}}
% Reference to a part, upper case
\def\Partref#1{Part~\ref{#1}}
\def\twopartref#1#2{parts \ref{#1} and \ref{#2}}

\def\ceil#1{\lceil #1 \rceil}
\def\floor#1{\lfloor #1 \rfloor}
\def\1{\bm{1}}
\newcommand{\train}{\mathcal{D}}
\newcommand{\valid}{\mathcal{D_{\mathrm{valid}}}}
\newcommand{\test}{\mathcal{D_{\mathrm{test}}}}

\def\eps{{\epsilon}}


% Random variables
\def\reta{{\textnormal{$\eta$}}}
\def\ra{{\textnormal{a}}}
\def\rb{{\textnormal{b}}}
\def\rc{{\textnormal{c}}}
\def\rd{{\textnormal{d}}}
\def\re{{\textnormal{e}}}
\def\rf{{\textnormal{f}}}
\def\rg{{\textnormal{g}}}
\def\rh{{\textnormal{h}}}
\def\ri{{\textnormal{i}}}
\def\rj{{\textnormal{j}}}
\def\rk{{\textnormal{k}}}
\def\rl{{\textnormal{l}}}
% rm is already a command, just don't name any random variables m
\def\rn{{\textnormal{n}}}
\def\ro{{\textnormal{o}}}
\def\rp{{\textnormal{p}}}
\def\rq{{\textnormal{q}}}
\def\rr{{\textnormal{r}}}
\def\rs{{\textnormal{s}}}
\def\rt{{\textnormal{t}}}
\def\ru{{\textnormal{u}}}
\def\rv{{\textnormal{v}}}
\def\rw{{\textnormal{w}}}
\def\rx{{\textnormal{x}}}
\def\ry{{\textnormal{y}}}
\def\rz{{\textnormal{z}}}

% Random vectors
\def\rvepsilon{{\mathbf{\epsilon}}}
\def\rvtheta{{\mathbf{\theta}}}
\def\rva{{\mathbf{a}}}
\def\rvb{{\mathbf{b}}}
\def\rvc{{\mathbf{c}}}
\def\rvd{{\mathbf{d}}}
\def\rve{{\mathbf{e}}}
\def\rvf{{\mathbf{f}}}
\def\rvg{{\mathbf{g}}}
\def\rvh{{\mathbf{h}}}
\def\rvu{{\mathbf{i}}}
\def\rvj{{\mathbf{j}}}
\def\rvk{{\mathbf{k}}}
\def\rvl{{\mathbf{l}}}
\def\rvm{{\mathbf{m}}}
\def\rvn{{\mathbf{n}}}
\def\rvo{{\mathbf{o}}}
\def\rvp{{\mathbf{p}}}
\def\rvq{{\mathbf{q}}}
\def\rvr{{\mathbf{r}}}
\def\rvs{{\mathbf{s}}}
\def\rvt{{\mathbf{t}}}
\def\rvu{{\mathbf{u}}}
\def\rvv{{\mathbf{v}}}
\def\rvw{{\mathbf{w}}}
\def\rvx{{\mathbf{x}}}
\def\rvy{{\mathbf{y}}}
\def\rvz{{\mathbf{z}}}

% Elements of random vectors
\def\erva{{\textnormal{a}}}
\def\ervb{{\textnormal{b}}}
\def\ervc{{\textnormal{c}}}
\def\ervd{{\textnormal{d}}}
\def\erve{{\textnormal{e}}}
\def\ervf{{\textnormal{f}}}
\def\ervg{{\textnormal{g}}}
\def\ervh{{\textnormal{h}}}
\def\ervi{{\textnormal{i}}}
\def\ervj{{\textnormal{j}}}
\def\ervk{{\textnormal{k}}}
\def\ervl{{\textnormal{l}}}
\def\ervm{{\textnormal{m}}}
\def\ervn{{\textnormal{n}}}
\def\ervo{{\textnormal{o}}}
\def\ervp{{\textnormal{p}}}
\def\ervq{{\textnormal{q}}}
\def\ervr{{\textnormal{r}}}
\def\ervs{{\textnormal{s}}}
\def\ervt{{\textnormal{t}}}
\def\ervu{{\textnormal{u}}}
\def\ervv{{\textnormal{v}}}
\def\ervw{{\textnormal{w}}}
\def\ervx{{\textnormal{x}}}
\def\ervy{{\textnormal{y}}}
\def\ervz{{\textnormal{z}}}

% Random matrices
\def\rmA{{\mathbf{A}}}
\def\rmB{{\mathbf{B}}}
\def\rmC{{\mathbf{C}}}
\def\rmD{{\mathbf{D}}}
\def\rmE{{\mathbf{E}}}
\def\rmF{{\mathbf{F}}}
\def\rmG{{\mathbf{G}}}
\def\rmH{{\mathbf{H}}}
\def\rmI{{\mathbf{I}}}
\def\rmJ{{\mathbf{J}}}
\def\rmK{{\mathbf{K}}}
\def\rmL{{\mathbf{L}}}
\def\rmM{{\mathbf{M}}}
\def\rmN{{\mathbf{N}}}
\def\rmO{{\mathbf{O}}}
\def\rmP{{\mathbf{P}}}
\def\rmQ{{\mathbf{Q}}}
\def\rmR{{\mathbf{R}}}
\def\rmS{{\mathbf{S}}}
\def\rmT{{\mathbf{T}}}
\def\rmU{{\mathbf{U}}}
\def\rmV{{\mathbf{V}}}
\def\rmW{{\mathbf{W}}}
\def\rmX{{\mathbf{X}}}
\def\rmY{{\mathbf{Y}}}
\def\rmZ{{\mathbf{Z}}}

% Elements of random matrices
\def\ermA{{\textnormal{A}}}
\def\ermB{{\textnormal{B}}}
\def\ermC{{\textnormal{C}}}
\def\ermD{{\textnormal{D}}}
\def\ermE{{\textnormal{E}}}
\def\ermF{{\textnormal{F}}}
\def\ermG{{\textnormal{G}}}
\def\ermH{{\textnormal{H}}}
\def\ermI{{\textnormal{I}}}
\def\ermJ{{\textnormal{J}}}
\def\ermK{{\textnormal{K}}}
\def\ermL{{\textnormal{L}}}
\def\ermM{{\textnormal{M}}}
\def\ermN{{\textnormal{N}}}
\def\ermO{{\textnormal{O}}}
\def\ermP{{\textnormal{P}}}
\def\ermQ{{\textnormal{Q}}}
\def\ermR{{\textnormal{R}}}
\def\ermS{{\textnormal{S}}}
\def\ermT{{\textnormal{T}}}
\def\ermU{{\textnormal{U}}}
\def\ermV{{\textnormal{V}}}
\def\ermW{{\textnormal{W}}}
\def\ermX{{\textnormal{X}}}
\def\ermY{{\textnormal{Y}}}
\def\ermZ{{\textnormal{Z}}}

% Vectors
\def\vzero{{\bm{0}}}
\def\vone{{\bm{1}}}
\def\vmu{{\bm{\mu}}}
\def\vtheta{{\bm{\theta}}}
\def\va{{\bm{a}}}
\def\vb{{\bm{b}}}
\def\vc{{\bm{c}}}
\def\vd{{\bm{d}}}
\def\ve{{\bm{e}}}
\def\vf{{\bm{f}}}
\def\vg{{\bm{g}}}
\def\vh{{\bm{h}}}
\def\vi{{\bm{i}}}
\def\vj{{\bm{j}}}
\def\vk{{\bm{k}}}
\def\vl{{\bm{l}}}
\def\vm{{\bm{m}}}
\def\vn{{\bm{n}}}
\def\vo{{\bm{o}}}
\def\vp{{\bm{p}}}
\def\vq{{\bm{q}}}
\def\vr{{\bm{r}}}
\def\vs{{\bm{s}}}
\def\vt{{\bm{t}}}
\def\vu{{\bm{u}}}
\def\vv{{\bm{v}}}
\def\vw{{\bm{w}}}
\def\vx{{\bm{x}}}
\def\vy{{\bm{y}}}
\def\vz{{\bm{z}}}

% Elements of vectors
\def\evalpha{{\alpha}}
\def\evbeta{{\beta}}
\def\evepsilon{{\epsilon}}
\def\evlambda{{\lambda}}
\def\evomega{{\omega}}
\def\evmu{{\mu}}
\def\evpsi{{\psi}}
\def\evsigma{{\sigma}}
\def\evtheta{{\theta}}
\def\eva{{a}}
\def\evb{{b}}
\def\evc{{c}}
\def\evd{{d}}
\def\eve{{e}}
\def\evf{{f}}
\def\evg{{g}}
\def\evh{{h}}
\def\evi{{i}}
\def\evj{{j}}
\def\evk{{k}}
\def\evl{{l}}
\def\evm{{m}}
\def\evn{{n}}
\def\evo{{o}}
\def\evp{{p}}
\def\evq{{q}}
\def\evr{{r}}
\def\evs{{s}}
\def\evt{{t}}
\def\evu{{u}}
\def\evv{{v}}
\def\evw{{w}}
\def\evx{{x}}
\def\evy{{y}}
\def\evz{{z}}

% Matrix
\def\mA{{\bm{A}}}
\def\mB{{\bm{B}}}
\def\mC{{\bm{C}}}
\def\mD{{\bm{D}}}
\def\mE{{\bm{E}}}
\def\mF{{\bm{F}}}
\def\mG{{\bm{G}}}
\def\mH{{\bm{H}}}
\def\mI{{\bm{I}}}
\def\mJ{{\bm{J}}}
\def\mK{{\bm{K}}}
\def\mL{{\bm{L}}}
\def\mM{{\bm{M}}}
\def\mN{{\bm{N}}}
\def\mO{{\bm{O}}}
\def\mP{{\bm{P}}}
\def\mQ{{\bm{Q}}}
\def\mR{{\bm{R}}}
\def\mS{{\bm{S}}}
\def\mT{{\bm{T}}}
\def\mU{{\bm{U}}}
\def\mV{{\bm{V}}}
\def\mW{{\bm{W}}}
\def\mX{{\bm{X}}}
\def\mY{{\bm{Y}}}
\def\mZ{{\bm{Z}}}
\def\mBeta{{\bm{\beta}}}
\def\mPhi{{\bm{\Phi}}}
\def\mLambda{{\bm{\Lambda}}}
\def\mSigma{{\bm{\Sigma}}}

% Tensor
\DeclareMathAlphabet{\mathsfit}{\encodingdefault}{\sfdefault}{m}{sl}
\SetMathAlphabet{\mathsfit}{bold}{\encodingdefault}{\sfdefault}{bx}{n}
\newcommand{\tens}[1]{\bm{\mathsfit{#1}}}
\def\tA{{\tens{A}}}
\def\tB{{\tens{B}}}
\def\tC{{\tens{C}}}
\def\tD{{\tens{D}}}
\def\tE{{\tens{E}}}
\def\tF{{\tens{F}}}
\def\tG{{\tens{G}}}
\def\tH{{\tens{H}}}
\def\tI{{\tens{I}}}
\def\tJ{{\tens{J}}}
\def\tK{{\tens{K}}}
\def\tL{{\tens{L}}}
\def\tM{{\tens{M}}}
\def\tN{{\tens{N}}}
\def\tO{{\tens{O}}}
\def\tP{{\tens{P}}}
\def\tQ{{\tens{Q}}}
\def\tR{{\tens{R}}}
\def\tS{{\tens{S}}}
\def\tT{{\tens{T}}}
\def\tU{{\tens{U}}}
\def\tV{{\tens{V}}}
\def\tW{{\tens{W}}}
\def\tX{{\tens{X}}}
\def\tY{{\tens{Y}}}
\def\tZ{{\tens{Z}}}


% Graph
\def\gA{{\mathcal{A}}}
\def\gB{{\mathcal{B}}}
\def\gC{{\mathcal{C}}}
\def\gD{{\mathcal{D}}}
\def\gE{{\mathcal{E}}}
\def\gF{{\mathcal{F}}}
\def\gG{{\mathcal{G}}}
\def\gH{{\mathcal{H}}}
\def\gI{{\mathcal{I}}}
\def\gJ{{\mathcal{J}}}
\def\gK{{\mathcal{K}}}
\def\gL{{\mathcal{L}}}
\def\gM{{\mathcal{M}}}
\def\gN{{\mathcal{N}}}
\def\gO{{\mathcal{O}}}
\def\gP{{\mathcal{P}}}
\def\gQ{{\mathcal{Q}}}
\def\gR{{\mathcal{R}}}
\def\gS{{\mathcal{S}}}
\def\gT{{\mathcal{T}}}
\def\gU{{\mathcal{U}}}
\def\gV{{\mathcal{V}}}
\def\gW{{\mathcal{W}}}
\def\gX{{\mathcal{X}}}
\def\gY{{\mathcal{Y}}}
\def\gZ{{\mathcal{Z}}}

% Sets
\def\sA{{\mathbb{A}}}
\def\sB{{\mathbb{B}}}
\def\sC{{\mathbb{C}}}
\def\sD{{\mathbb{D}}}
% Don't use a set called E, because this would be the same as our symbol
% for expectation.
\def\sF{{\mathbb{F}}}
\def\sG{{\mathbb{G}}}
\def\sH{{\mathbb{H}}}
\def\sI{{\mathbb{I}}}
\def\sJ{{\mathbb{J}}}
\def\sK{{\mathbb{K}}}
\def\sL{{\mathbb{L}}}
\def\sM{{\mathbb{M}}}
\def\sN{{\mathbb{N}}}
\def\sO{{\mathbb{O}}}
\def\sP{{\mathbb{P}}}
\def\sQ{{\mathbb{Q}}}
\def\sR{{\mathbb{R}}}
\def\sS{{\mathbb{S}}}
\def\sT{{\mathbb{T}}}
\def\sU{{\mathbb{U}}}
\def\sV{{\mathbb{V}}}
\def\sW{{\mathbb{W}}}
\def\sX{{\mathbb{X}}}
\def\sY{{\mathbb{Y}}}
\def\sZ{{\mathbb{Z}}}

% Entries of a matrix
\def\emLambda{{\Lambda}}
\def\emA{{A}}
\def\emB{{B}}
\def\emC{{C}}
\def\emD{{D}}
\def\emE{{E}}
\def\emF{{F}}
\def\emG{{G}}
\def\emH{{H}}
\def\emI{{I}}
\def\emJ{{J}}
\def\emK{{K}}
\def\emL{{L}}
\def\emM{{M}}
\def\emN{{N}}
\def\emO{{O}}
\def\emP{{P}}
\def\emQ{{Q}}
\def\emR{{R}}
\def\emS{{S}}
\def\emT{{T}}
\def\emU{{U}}
\def\emV{{V}}
\def\emW{{W}}
\def\emX{{X}}
\def\emY{{Y}}
\def\emZ{{Z}}
\def\emSigma{{\Sigma}}

% entries of a tensor
% Same font as tensor, without \bm wrapper
\newcommand{\etens}[1]{\mathsfit{#1}}
\def\etLambda{{\etens{\Lambda}}}
\def\etA{{\etens{A}}}
\def\etB{{\etens{B}}}
\def\etC{{\etens{C}}}
\def\etD{{\etens{D}}}
\def\etE{{\etens{E}}}
\def\etF{{\etens{F}}}
\def\etG{{\etens{G}}}
\def\etH{{\etens{H}}}
\def\etI{{\etens{I}}}
\def\etJ{{\etens{J}}}
\def\etK{{\etens{K}}}
\def\etL{{\etens{L}}}
\def\etM{{\etens{M}}}
\def\etN{{\etens{N}}}
\def\etO{{\etens{O}}}
\def\etP{{\etens{P}}}
\def\etQ{{\etens{Q}}}
\def\etR{{\etens{R}}}
\def\etS{{\etens{S}}}
\def\etT{{\etens{T}}}
\def\etU{{\etens{U}}}
\def\etV{{\etens{V}}}
\def\etW{{\etens{W}}}
\def\etX{{\etens{X}}}
\def\etY{{\etens{Y}}}
\def\etZ{{\etens{Z}}}

% The true underlying data generating distribution
\newcommand{\pdata}{p_{\rm{data}}}
% The empirical distribution defined by the training set
\newcommand{\ptrain}{\hat{p}_{\rm{data}}}
\newcommand{\Ptrain}{\hat{P}_{\rm{data}}}
% The model distribution
\newcommand{\pmodel}{p_{\rm{model}}}
\newcommand{\Pmodel}{P_{\rm{model}}}
\newcommand{\ptildemodel}{\tilde{p}_{\rm{model}}}
% Stochastic autoencoder distributions
\newcommand{\pencode}{p_{\rm{encoder}}}
\newcommand{\pdecode}{p_{\rm{decoder}}}
\newcommand{\precons}{p_{\rm{reconstruct}}}

\newcommand{\laplace}{\mathrm{Laplace}} % Laplace distribution

\newcommand{\E}{\mathbb{E}}
\newcommand{\Ls}{\mathcal{L}}
\newcommand{\R}{\mathbb{R}}
\newcommand{\emp}{\tilde{p}}
\newcommand{\lr}{\alpha}
\newcommand{\reg}{\lambda}
\newcommand{\rect}{\mathrm{rectifier}}
\newcommand{\softmax}{\mathrm{softmax}}
\newcommand{\sigmoid}{\sigma}
\newcommand{\softplus}{\zeta}
\newcommand{\KL}{D_{\mathrm{KL}}}
\newcommand{\Var}{\mathrm{Var}}
\newcommand{\standarderror}{\mathrm{SE}}
\newcommand{\Cov}{\mathrm{Cov}}
% Wolfram Mathworld says $L^2$ is for function spaces and $\ell^2$ is for vectors
% But then they seem to use $L^2$ for vectors throughout the site, and so does
% wikipedia.
\newcommand{\normlzero}{L^0}
\newcommand{\normlone}{L^1}
\newcommand{\normltwo}{L^2}
\newcommand{\normlp}{L^p}
\newcommand{\normmax}{L^\infty}

\newcommand{\parents}{Pa} % See usage in notation.tex. Chosen to match Daphne's book.

\DeclareMathOperator*{\argmax}{arg\,max}
\DeclareMathOperator*{\argmin}{arg\,min}

\DeclareMathOperator{\sign}{sign}
\DeclareMathOperator{\Tr}{Tr}
\let\ab\allowbreak



\usepackage{hyperref}
\usepackage{cleveref}
\usepackage{url}
\usepackage{multirow}
%\usepackage{enquote}

\title{Replay-based continual learning with constant time complexity}

% Authors must not appear in the submitted version. They should be hidden
% as long as the \iclrfinalcopy macro remains commented out below.
% Non-anonymous submissions will be rejected without review.

\author{Alexander Krawczyk and Alexander Gepperth \thanks{\url{www.gepperth.net/alexander}} \\
Department of Applied Computer Science\\
University of Applied Sciences Fulda\\
Leipziger Str. 123, 36037 Fulda, Germany \\
\texttt{alexander.\{krawczyk,gepperth\}@cs.hs-fulda.de} \\
}

% The \author macro works with any number of authors. There are two commands
% used to separate the names and addresses of multiple authors: \And and \AND.
%
% Using \And between authors leaves it to \LaTeX{} to determine where to break
% the lines. Using \AND forces a linebreak at that point. So, if \LaTeX{}
% puts 3 of 4 authors names on the first line, and the last on the second
% line, try using \AND instead of \And before the third author name.

\newcommand{\fix}{\marginpar{FIX}}
\newcommand{\new}{\marginpar{NEW}}
\newcommand{\vgamma}{{\bm \gamma}}
\newcommand{\enquote}[1]{"#1"}

%\iclrfinalcopy % Uncomment for camera-ready version, but NOT for submission.
\begin{document}


\maketitle

\begin{abstract}
We propose a new constant-time approach to replay-based continual learning (CL). 
%
GR is a strategy that protects existing knowledge from previous and inaccessible training sessions (sub-tasks) by having auxilliary generator networks produce samples from those sub-tasks and merging them with the current sub-task.
%
The main innovation we propose is twofold: first of all, we propose \textbf{selective modification} of existing knowledge only where it overlaps with the current sub-task. On the other hand,
we propose \textbf{selective replay} of samples, to protect existing knowlege in opverlapping areas only. 

Since only the overlapping parts of previous knowledge need to be protected against CF, fewer samples can be replayed. We use this to  maintain a constant ratio between real and generated samples, irrespectively of the number of processed sub-tasks. Consequently, training time is kept constant for all sub-tasks.
%
We test the proposed strategy on CL problems derived from visual classification and reinforcement learning problems, and present a comparison to VAE-based generative replay.
\end{abstract}

\section{Introduction}
\begin{comment}
PLAN:
variant generation demonstrieren mit flat 2-laer dcgmm. mnist, svhn
incremental property von gmms demonstrieren: nur betroffene gebiete werden ersetzt!
varint generation demo mit VAE auf mnist, SVHN. aber warum fkt das nicht --> vaes sind nicht incremental, man müsste alles abspielen um sie auf stand zu bringen
focus on replay, not on absolute per.
compare to genrep und show time complexity of learning for 3-tasks
show that perf on past tasks does not degrade most for 9-1 but also ok for 1-9, cf nadzeya
assumptiomn: only small extension of existing knowledge at each task
\end{comment}
% Context: CL
% context: genrep, marginal conditional replay. cite successes, ...
% motivation: discuss inear dependency, give fromula and graph. problems:
% a) T-1 dependency for generated samples b) all classes quiprobable c)unbalanced distr for marginal replay??
% d) smallest additions --> biggest replay! cite nadzeya
% goals/contribution: 
% relwork
% basic approach: 
% ---
% methods, data!
% discussion
%
% context
This contribution is in the context of continual learning (CL), a recent flavor of machine learning that investigates learning from data with a non-stationary distribution.
A common effect in this context is catastrophic forgetting (CS), an effect where previously acquired knowledge is abruptly lost after a change in the data distribution. 
In the default scenario for CL (see, e.g., \cite{bagus22b}), a number of assumptions are made in order to render CL more tractable: first of all, distribution changes are assumed to be abrupt, partitioning the data stream into \textit{sub-tasks} of stationary distribution. Then, sub-task onsets are supposed to be known, instead of inferring them from data. And lastly, sub-tasks are assumed to be disjoint, i.e., not containing the same classes in supervised scenarios. Please refer to \cref{fig:default} for a visualization of the default scenario. Together with this goes the constraint that no, or only a few, samples may be stored.
%
\begin{figure}[b]
    \centering
    \includegraphics*[width=0.7\textwidth,page=4,viewport=0cm 0cm 14.3cm 5cm]{figs.pdf}
    \caption{The default scenario for (supervised) CL. The data stream is assumed to be partitioned into \textit{sub-tasks} $T_i$. Statistics within a sub-task are considered stationary, and sub-task data and labels (targets) are assumed to be disjoint, i.e., from different classes (MNIST classes used here for visualization). Sub-task onsets are assumed to be known as well.
    \label{fig:default}
    }
\end{figure}
%
% context: GR
A very promising line of work on CL in the default scenario are rehearsal methods. Rehearsal aims at preventing CF by using samples from previous sub-tasks to augment the current one. On the one hand, there are \enquote{true} rehearsal methods which use a small number of stored samples for augmentation. On the other hand, there are \textit{pseudo-rehearsal} or \textit{replay} methods, where the samples to augment the current sub-task are produced in unlimited number by a generator, which removes the need to store samples. A schematics of the training process in generative replay is given in \cref{fig:genrep}.
\begin{figure}[t]
    \centering
    \includegraphics*[height=3cm,page=1,viewport=0in 0in 7.5in 2in]{figs.pdf}
    \caption{In generative replay, a \textit{scholar} is trained at every sub-task. Scholars are composed of generator and solver modules. The solver performs the task, e.g., classification, whereas the generator serves as a memory for samples from previous sub-tasks $T_{i'}$, $i' < i$. Please note that the amount of generated data usually far exceeds the amount of new data. }
    \label{fig:genrep}
\end{figure}
%
% Problems with CL 
As promising as replay seems to be as a principled approach to CL, it presents several challenges: 
First of all, if DNNs are employed as solvers and generators, then all classes must be represented in equal proportion in the training data at every sub-task, since DNNs are sensible to class frequencies. 
For the simple case of a single new class of $D$ samples per sub-task, the generator must produce $(i-1)D$ samples at sub-task $T_i$ in order to always train with $D$ samples per class. More generally, class balancing requires $\sum_{j=1}^{i-1} D_j$ samples, if $D_j$ indicates the number of samples at sub-task $T_j$ to be replayed at sub-task $T_i$.
By this linear dependency, even very small additions to a large body of existing knowledge (a common use-case) require a large amount of samples for replay, see, e.g., \cite{nadz22}. Since replay is always a lossy process, this imposes a severe limit on GR-based CL performance.
%
%Furthermore, DNN solvers and generators must be re-trained \textit{from scratch} at every sub-task: existing weight configurations are discarded instead of re-used to speed up training. 
%In addition, we are at the mercy of the generator to produce samples in equal frequencies, although this can me mitigated somewhat by class-conditional generation of samples \cite{timalex}.
Lastly, 
%
\subsection{Approach}
%
We propose a replay-based approach ((visualized in \cref{fig:var}) relying on fully probabilistic models of machine learning, namely Gaussian Mixture Models (GMMs) and corresponding convolutional extensions introduced in \cite{gmmsgd,dcgmm-ijcnn}.
%In contrast to DNNs where knowledge is represented in a completely delocalized fashion, GMMs implement a semi-localized knowledge representation.
A key concept is an explicit similarity (probability) measure defined by centroids (prototypes) and covariance matrices, which can be used to compare incoming data to existing knowledge, represented by GMM \textit{components}. 
As shown in \cite{sgdgmm}, only the GMM components that are similar to incoming data are adapted. Conversely, knowledge represented by dissimilar components is unaffected and thus protected against CF. 
Incoming data are augmented by sampling from protoypes in overlapping areas, i.e., prototypes with highest similarity, see \cref{fig:approach}, which ensures that the prototypes are adapted but not overwritten. 
\begin{figure}
    \centering
    \includegraphics*[width=0.7\textwidth,page=3,viewport=0in 0in 12cm 3cm]{figs.pdf}
    \caption{The essence of the proposed ReCT-CL approach, represented in data space. At every sub-task $T_i$, data with new statistics (red area) is presented to a scholar which has acquired knowledge of previous sub-tasks (blue area), which is encoded by GMM \textit{components} (yellow crosses), see text for details. The diagrams represent the situation before (left) and after (right) training on sub-task $T_i$. Training data for $T_i$ is created by a union of samples from $T_i$ and samples generated from the components in the overlapping area. After training, components in this area have shifted to represent the new statistics as well. }
    \label{fig:approach}

\end{figure}
%
\subsection{Contributions}
The salient points of ReCT-CL are:
%
\par\noindent\textbf{Selective replay:} existing knowledge is not replayed indiscriminately, but only where significant overlap with new data exists. The detection of overlaps is an intrinsic property of the employed GMMs.
%
\par\noindent\textbf{Selective replacement:} The scholar is not re-initialized between sub-tasks, and existing knowledge is only 
modified where an overlap with new data exists. Selective replacement is an intrinsic property of GMMs.
%
\par\noindent\textbf{Constant time complexity:} Since only a small part of existing knowledge needs to be considered for replay at each sub-task, the number of generated/replayed samples can be small as well. The ratio between new and generated data can therefore be fixed, irrespectectively of past sub-tasks.
%
\subsection{Related work}


\section{Methods}
\label{sec:methods}
%
\subsection{Data}\label{sec:data}
% MNIST, FMNIST, RL-data
\noindent For the evaluation we use the following image datasets:
\smallskip\\
\noindent\textbf{MNIST}~\cite{LeCun1998} consists of $60\,000$ $28$\,$\times$\,$28$ gray scale images of handwritten digits (0-9).\\
\noindent\textbf{FashionMNIST}~\cite{Xiao2017} consists of images of clothes in 10 categories and is structured like MNIST. \\
\noindent\textbf{CRL}~\cite{bagus2022b} contains 30000 mono images from an RL task performed by a simulated robot. Each of the $15$\,$\times$\,$100$) images
is formed by concatenating line camera images from the last 3 RL time steps.

These datasets are not particularly challenging w.r.t. non-continual classification, although CL tasks formed from these datasets according to the default scenario (\cref{sec:intro:default}) have been shown to be very difficult, see, e.g., \cite{clstuff}. Particular examples of CL tasks formed from MNIST and FashionMNIST that have been used in the literature are $D5-5$, $D9-1$, $D2^5$ or $D1^{10}$, where each number indicates the amount of classes contained in each sub-task. 
%
\subsection{ReCT-CL fundamentals}\label{sec:gmm}
%
\begin{figure}[b]
    \centering
    \includegraphics*[height=3cm,page=2,viewport=0in 0in 6in 2in]{figs.pdf}
    \caption{The proposed ReCT-CL approach. The main difference to conventional generative replay is that new data are used to \textit{query} the current scholar for similar samples. The ratio of generated and new data can thus be chosen constant as explained in the text. The scholar is not reinitialized but re-trained from its current state with generated and new data.}
    \label{fig:var}
\end{figure}
%
% big picture I
In contrast to conventional replay, where a scholar is composed of a generator and a solver network (see \cref{fig:gr}), ReCT-CL proposes a scholar where single network performs both functions. This network is composed of a Gaussian Mixture Model (GMM) layer followed by a linear regression readout layer. All layers are trained by SGD according to \cite{sgdgmm}. For improving classification and generation performance, we have also used deep convolutional GMMs (DCGMMs) as described in \cref{dcgmm}, which can contain a succession of convolutional GMM, half-convolution and pooling layers. The mathematical treatment of ReCT-CL scholars will assume a single GMM layer only without loss of generality. 
\\
% GMMs
GMMs describe data distributions by a set of $K$ \textit{components}, consisting of component weights $\pi_k$, centroids $\vmu_k$ and covariance matrices $\mSigma_k$. A data sample $\vx$ is assigned a probability $p(\vx) = \sum_k \pi_k \mathcal N(\vx ; \vmu_k, \mSigma_k)$ as a weighted sum of normal distributions $\mathcal N(\vx; \vmu_k, \mSigma_k)$. Training of GMMs is performed as detailed in \cite{sgdgmm} by adapting centroids, covariance matrices and component weights through the SGD-based minimization of the log-likelihood $\mathcal L=\sum_n \log \sum_k \pi_k \mathcal N(\vx_n; \vmu_k,\mSigma_k)$.
\\
% classification and vartiant gen
The GMMs employed in ReCT-CL have two basic functions: classification and variant generation.
For classification, 
we first compute the GMM \textit{responsibilities} $\gamma_k(\vx_n) = \frac{\pi_k\mathcal N(\vx_n; \vmu_k,\mSigma_k)}{\sum_j \pi_j \mathcal N(\vx_n; \vmu_j,\mSigma_j)}$ and feed them into a linear regression layer as $\vo(\vx_n) = \mW\vgamma(\vx_n) + \vb$.
Variant generation is a special case of GMM sampling: given a data sample $\vx_n$, we first determine the closest component: $i = \text{argmax}_k \gamma_k(\vx_n)$ and then draw a sample from the normal distribution defined by that component: $\hat \vx \sim \mathcal N(\cdot; \vmu_i, \mSigma_i)$.
\\
% suitability for CL training
To ensure CL capacity, ReCT-CL training relies on three key concepts: loss masking, MSE classification, and max-component approximation for GMMs. 
Loss masking is a technique that assigns different learning rates $\epsilon_n$ to individual samples $\vec x_n$, depending on whether they are classified correctly ($\chi_n = 1$) or not ($\chi_n = 0$): 
$\epsilon_n = \chi_n\epsilon_\uparrow + (1-\chi_n)\epsilon_{\downarrow}$.
In order to make the classification layer suitable for CL, we employ an MSE loss instead of cross-entropy for target values $\vt_n$ and model outputs $vo(\vx_n)$: $\mathcal L_\text{MSE} = \sum_n ||\vo(\vx_n)-\vt_n||^2$. % REASON??
Lastly, we use the annealing procedure and max-component approximation to the GMM log-likelihood described in \cref{sgdgmm} for training the GMM layer. This ensures that only best-matching (most similar) component is adapted (plus nearby components, at least during early training stages), whereas dissimilar components are unaffected. The neighbourhood of affected components around the best-matching one is defined by the annealing radius $r(t)$ which is automatically reduced over time, see \cite{sgdgmm} for details.
\\
% high-level
\Cref{fig:var} shows the high-level logic of ReCT-CL. We assume that unknown data from a new sub-task arrives in a batch. 
Each sample of that batch is used to \textit{query} the GMM for a similar, known sample using variant generation. Mixing new and generated samples in a defined, constant proportion creates the training data for the current sub-task. 
This is the driving idea behind ReCT-CL: 
a new sample causes adaptation in its best-matching component. However, variants generated by that sample will almost always cause adaptation to the same component. The component will therefore converge to a compromise between new data and existing knowledge, while
dissimilar components stay completely unaffected. 
%

\subsection{ReCTL-CL implementation}
% ReCTL-CL model description
% TODO: Müssen wir erklären wie der Klassifizierer von den Prototypen Aktivierungen der lokalen Patches profitiert ? %
The ReCTL-CL implementation consists of two separate, running in parallel GMM layers. GMM $L_{S}$ with $K=100$ components operates on the global input image and functions as the sampling layer for variant generation. GMM $L_{C}$, with $K=49$ components, operates on local image patches from a upstream half padded convolution with a kernel size of $8x8$ and a stride $s=2$. The main objective of the $L_{CL}$ is to improve the overall classification results. The output, namely, the GMM responsibilities of both layers get concatenated and are presented to the linear regression as an final input.
% GMM & LR Hyperparams
Hyperparameters for both ReCTL-CL GMM layers are task-independent and the free parameters, as well as the initialization values are based on empirically well working values presented in \cite{sgdgmm}. The annealing control default parameters from \cite{sgdgmm} are used, and the learning rates for both GMM layers were set to $\epsilon=0.01$, while the linear regression used $\eps_{LR}=0.05$. ReCT-CL optimization relies on plain SGD since Adam seems to have problems optimizing GMMs by SGD. Both, the ReCTL-CL GMM layers and the linear regression were trained simultaneously for each training task. The initial annealing radius is set to $r_0=\sqrt{0.125} K$, and we reset the annealing radius to a constant value of $r_{replay}=0.25$ each time before starting with a new replay-task.
%
We provide a publicly available TensorFlow 2 implementation\footnote{the public repository can be accessed via the following \href{https://github.com/TODO/TODO}{\textcolor{blue}{link}}} to reconstruct the experiments on a local machine.
% TODO: Falls Loss-Masking verwendet:
	% For the initial sub-task $T_1$, loss masking is disables and
	% For sub-tasks $T_{i>1}$, loss masking is enabled for all layers
	% loss masking parameters of $\epsilon_\downarrow = 0.01$ and $\epsilon_\uparrow=1$. 
%
\subsection{VAE-based generative replay implementation}
%
\begin{table*}
\caption{DNN structure for VAE-based replay.
\label{tab:networkstructure}
}
\centering
\tiny
\begin{tabular}{|c|c|}
\hline
\textbf{Component}    &  \textbf{Structure}  \\
\hline
VAE Encoder  & Conv2D(32,5,2)-ReLU-Conv2D(64,5,2)-ReLU-Dense(100)-\\
               &  ReLU-Dense(50)-ReLU-Dense(25) \\
VAE Decoder & Dense(100)-ReLU-Dense(3136)-Reshape(16,14,14)-ReLU-\\
            & Conv2DTranspose(32,5,2)-RelU-Conv2DTranspose(1,5,2)-Sigmoid \\
Solver      & Conv2D(32,5,1)-MaxPool2D(2)-ReLU-Conv2D(64,5,1)-MaxPool2D(2)- \\      
            & ReLU-Dense(100)-ReLU-Dense(10)\\
\hline
\end{tabular}
\end{table*}
%
%
We perform generative replay using VAEs as generators as described in \cite{nadz22}, 
with a VAE latent dimension of 25, a disentangling factor $\beta=1.$ and conditional generation turned off.
The structure of the VAE generator and the solver is given in \cref{tab:networkstructure}.
%
The learning rate for solvers and VAEs to $\epsilon_S=10^{-4}$, $\epsilon_{VAE}=10^{-3}$.
VAE and solver optimization is conducted using the Adam optimizer, 
%
\subsection{Performance measures}\label{sec:exppeval}

\section{Experiments}\label{sec:exp}
% 
\subsection{Experimental settings}
% General experimental setup, sub-tasks, baseline experiments
The experiments evaluate each models' performance in a class-incremental learning scenario. 
The sub-tasks $T_{i>1}$, were selected to be closer to a practical application scenario, since ML applications \"in the wild\" often operate in streaming scenarios. 
%
With our experiments we emphasize the case, where novel classes are explored in a sequential manner, while assuming that novel data to be learned from, only represents a smaller fraction to be added to the existing knowledge of a scholar $S_{i>1}$. 
%
The class combinations of the initial training and (replay) sub-tasks examined, are showcased in Tab. \cref{tab:slts}. 
\\
Before performing replay training, a initial run on $T_1$ is executed. This is followed by a sequence of $i>1$ independent (replay) sub-tasks $T_{i>1}$, each containing samples of a single and yet unseen class of the dataset. The goal is to allow a integration of new knowledge to the existing scholar without destroying information obtained in past training tasks.
%
Baseline experiments for MNIST and FashionMNIST were conducted in a non CL setting, where the models have been trained on all available classes at once.
%
\begin{table*}
\caption{MNIST and FashionMNIST task splits. Each tasks contains the samples and labels from classes corresponding to the integers.  $T_1$ denotes the initial training task, while  $T_i>1$ denote the replay training tasks. 
\label{tab:slts}
}
\centering
\begin{tabular}{ c | c | c | c | c }
	task name & $T_1$ & $T_2$ & $T_3$ & $T_4$ \\ [0.5ex]
	\hline
	8-1-1a  	& [0-7] 	& 8 		& 9 		& 		\\
	8-1-1b  	& [2-9] 	& 0 		& 1 		&  		\\
	7-1-1-1a    & [0-6] 	& 7 		& 8 		& 9		\\
 	7-1-1-1b    & [3-9] 	& 0 		& 1 		& 2 
\end{tabular}
\end{table*}
%
We perform 10 independently and randomly initialized runs for all VAE replay and ReCT-CL experiments, produce the appropriate metrics (see \cref{sec:exppeval}) and average these over all runs.
%
After each sub-task $T_{i>0}$, the resulting scholar $S_i$ is evaluated by measuring the classification accuracy of the solver. 
%
Evaluation is done over 1 epoch on each sub-tasks' held out test set. An additional test on a joint test set $D_{ALL}$ (containing all classes) was also measured.
%
Training is performed on mini-batches with size $\beta=100$, where each mini-batch for replay training $T_{i>1}$ is formed by mixing generated samples $\beta_G$ from the current scholar's generator instance $S_{i>1}$ with real data $\beta_R$ from the actual training set. 
%
The mixing ratio of generated to real data follows a proportional strategy, which is based on the amount of classes already encountered (e.g., if a scholar learned about 9 classes, and is trained on a sub-task containing samples from a yet unseen class, the corresponding mini-batches are assembled by $\beta_G = 90$, and $\beta_R=10$).
% VAE-specific
At each sub-task, the VAE solvers are trained for 10 epochs on new and generated data, whereas VAE generators are trained for 50 epochs.
VAE experiments are conducted in two settings: \textbf{constant-time} and \textbf{balanced}. In the constant-time setting, 
we generate $D_i$ samples for replay if there are $D_i$ samples in a new sub-task $T_i$. In the balanced setting, the number of generated samples is computed as $\sum_{j=1}^{i-1} D_j$ in order to achieve balanced classes, see \cref{sec:intro}.
ReCT-CL experiments are conducted in the constant-time setting only.
%
% GMM-specific
% TODO: Falls zusätzliches Training oder Warten mit Regressor-Training hilft, ÄNDERN!!!
The ReCTL-CL training follows the annealing regularization (see. \cite{sgdgmm}), and implements an early stoppage mechanism, terminating training as soon as GMM layers $L_S, L_C$ reached a plateau of stationary training loss for the current sub-task $T_i$.

\subsection{Intrinsic CL capacity of GMMs}
\subsection{Variant generation with GMMs and DCGMMs}
\subsection{Model comparison at variable time complexity}
\begin{table*}
\caption{ReCT-CL and VAE performance indicators for FashionMNIST tasks.
\label{tab:expres_rectcl_fmnist}
}
\centering
\begin{tabular}{| c c | c c c c || c c c c |}
    \hline
    \multicolumn{10}{|c|}{\textbf{FashionMNIST}} \\
	\hline
	& & \multicolumn{8}{|c|}{\textbf{class. acc. for $D_i$ in \% after training on $T_i$}}       \\
	& & \multicolumn{4}{|c||}{\textbf{ReCT-CL}} & \multicolumn{4}{|c|}{\textbf{VAE-DGR} $O(n)$} \\
	\textbf{task} & \textbf{tested on}      & $T_1$ & $T_2$ & $T_3$ & $T_4$ & $T_1$ & $T_2$ & $T_3$ & $T_4$ \\
	\hline
	\textbf{baseline}                       & $D_{ALL}$ & 00.0 & / & / & /          & 00.0 & / & / & /       \\
	\hline
	\multirow{3}{3.25em}{\textbf{8-1-1a}}   & $D_1$     & 00.0 & 00.0 & 00.0 & /    & 00.0 & 00.0 & 00.0 & / \\
	                                        & $D_2$     & 00.0 & 00.0 & 00.0 & /    & 00.0 & 00.0 & 00.0 & / \\
	                                        & $D_3$     & 00.0 & 00.0 & 00.0 & /    & 00.0 & 00.0 & 00.0 & / \\
	                                        & $D_{ALL}$ & 00.0 & 00.0 & 00.0 & /    & 00.0 & 00.0 & 00.0 & / \\
	\hline
	\multirow{3}{3.25em}{\textbf{8-1-1b}}   & $D_1$     & 00.0 & 00.0 & 00.0 & /    & 00.0 & 00.0 & 00.0 & / \\
	                                        & $D_2$     & 00.0 & 00.0 & 00.0 & /    & 00.0 & 00.0 & 00.0 & / \\
	                                        & $D_3$     & 00.0 & 00.0 & 00.0 & /    & 00.0 & 00.0 & 00.0 & / \\
	                                        & $D_{ALL}$ & 00.0 & 00.0 & 00.0 & /    & 00.0 & 00.0 & 00.0 & / \\
	\hline
	\multirow{3}{3.25em}{\textbf{7-1-1a}}   & $D_1$     & 00.0 & 00.0 & 00.0 & 00.0     & 00.0 & 00.0 & 00.0 & 00.0 \\
	                                        & $D_2$     & 00.0 & 00.0 & 00.0 & 00.0     & 00.0 & 00.0 & 00.0 & 00.0 \\
	                                        & $D_3$     & 00.0 & 00.0 & 00.0 & 00.0     & 00.0 & 00.0 & 00.0 & 00.0 \\
	                                        & $D_4$     & 00.0 & 00.0 & 00.0 & 00.0     & 00.0 & 00.0 & 00.0 & 00.0 \\
	                                        & $D_{ALL}$ & 00.0 & 00.0 & 00.0 & 00.0     & 00.0 & 00.0 & 00.0 & 00.0 \\
	\hline
	\multirow{3}{3.25em}{\textbf{7-1-1a}}   & $D_1$     & 00.0 & 00.0 & 00.0 & 00.0     & 00.0 & 00.0 & 00.0 & 00.0 \\
	                                        & $D_2$     & 00.0 & 00.0 & 00.0 & 00.0     & 00.0 & 00.0 & 00.0 & 00.0 \\
	                                        & $D_3$     & 00.0 & 00.0 & 00.0 & 00.0     & 00.0 & 00.0 & 00.0 & 00.0 \\
	                                        & $D_4$     & 00.0 & 00.0 & 00.0 & 00.0     & 00.0 & 00.0 & 00.0 & 00.0 \\
	                                        & $D_{ALL}$ & 00.0 & 00.0 & 00.0 & 00.0     & 00.0 & 00.0 & 00.0 & 00.0 \\
	\hline
\end{tabular}
\end{table*}
\subsection{Model comparison at constant time complexity}
\subsection{Generalization: ReCT-CL using DCGMMs}

\section{Discussion and conclusion}
% Assumptions: a) overlap is small compared to existzing knowledge. Gets more reasonable as more subtasks are added
% impossible to measure how much knowledge a new sub-task contribnutes. Nr samples !=v amount of knowledge.
% how to balance in VAE-replay?? based on nr of samples only!
% outlier detection possible as well!
% why not GANS?
% clear that ReCT-CL is not super performant yet, but metrics is constant time complexity. See review nrw measures ...
%
\subsubsection*{Author Contributions}
AK created the library used for ReCT-CL experiments, based on earlier code by AG. AG wrote a significant part of the manuscript and designed the experiments. AK conducted he GMM-based experiments, AG the VAE-based ones. Both contributed to the evaluation and the description of the experiments in the manuscript, as well as to proof-reading.
%
%\subsubsection*{Acknowledgments}
%Use unnumbered third level headings for the acknowledgments. All
%acknowledgments, including those to funding agencies, go at the end of the paper.
%
\bibliographystyle{plain}
\bibliography{CRL,continual}

\appendix
\section{Appendix}
You may include other additional sections here.

\end{document}
